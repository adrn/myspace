% This file is part of the myspace project.
% Copyright 2019 the authors. All rights reserved.

% TODO:
% -

\documentclass[modern]{aastex62}

\usepackage{amsmath}

% typography
\setlength{\parindent}{1.\baselineskip}
% Load common packages
% \usepackage{microtype}  % ALWAYS!
\usepackage{amsmath}
\usepackage{amsfonts}
\usepackage{amssymb}
\usepackage{booktabs}

\usepackage{graphicx}
\usepackage{color}

% \definecolor{cbblue}{HTML}{3182bd}
% \usepackage{hyperref}
% \definecolor{linkcolor}{rgb}{0.02,0.35,0.55}
% \definecolor{citecolor}{rgb}{0.45,0.45,0.45}
% \hypersetup{colorlinks=true,linkcolor=linkcolor,citecolor=citecolor,
%             filecolor=linkcolor,urlcolor=linkcolor}
% \hypersetup{pageanchor=true}

\newcommand{\documentname}{\textsl{Article}}
\newcommand{\sectionname}{Section}
\renewcommand{\figurename}{Figure}
\newcommand{\equationname}{Equation}
\renewcommand{\tablename}{Table}

% Packages / projects / programming
\newcommand{\package}[1]{\textsl{#1}}
\newcommand{\acronym}[1]{{\small{#1}}}
\newcommand{\github}{\package{GitHub}}
\newcommand{\python}{\package{Python}}
\newcommand{\emcee}{\package{emcee}}

% Missions
\newcommand{\project}[1]{\textsl{#1}}
\newcommand{\gaia}{\project{Gaia}}
\newcommand{\pans}{\project{Pan-STARRS}}

% For referee
\newcommand{\changes}[1]{{\color{red} #1}}
\definecolor{mahogany}{RGB}{165,15,21}
\newcommand{\resp}[1]{{\color{mahogany}#1}}

% Stats / probability
\newcommand{\given}{\,|\,}
\newcommand{\norm}{\mathcal{N}}

% Maths
\newcommand{\dd}{\mathrm{d}}
\newcommand{\transpose}[1]{{#1}^{\mathsf{T}}}
\newcommand{\inverse}[1]{{#1}^{-1}}
\newcommand{\argmin}{\operatornamewithlimits{argmin}}
\newcommand{\mean}[1]{\left< #1 \right>}

% Unit shortcuts
\newcommand{\msun}{\ensuremath{\mathrm{M}_\odot}}
\newcommand{\kms}{\ensuremath{\mathrm{km}~\mathrm{s}^{-1}}}
\newcommand{\mps}{\ensuremath{\mathrm{m}~\mathrm{s}^{-1}}}
\newcommand{\pc}{\ensuremath{\mathrm{pc}}}
\newcommand{\kpc}{\ensuremath{\mathrm{kpc}}}
\newcommand{\kmskpc}{\ensuremath{\mathrm{km}~\mathrm{s}^{-1}~\mathrm{kpc}^{-1}}}

% Misc.
\newcommand{\bs}[1]{\boldsymbol{#1}}

% Astronomy
\newcommand{\DM}{{\rm DM}}
\newcommand{\feh}{\ensuremath{{[{\rm Fe}/{\rm H}]}}}
\newcommand{\df}{\acronym{DF}}

% TO DO
\newcommand{\todo}[1]{{\color{red} TODO: #1}}


% aastex parameters
% \received{not yet; THIS IS A DRAFT}
%\revised{not yet}
%\accepted{not yet}
% % Adds "Submitted to " the arguement.
% \submitjournal{ApJ}
\shorttitle{TODO}
\shortauthors{Price-Whelan \& Hogg}

\begin{document}\sloppy\sloppypar\raggedbottom\frenchspacing % trust me

\title{myspace}

\author[0000-0003-0872-7098]{Adrian~M.~Price-Whelan}
\affiliation{Department of Astrophysical Sciences,
             Princeton University, Princeton, NJ 08544, USA}
\email{adrn@astro.princeton.edu}
\correspondingauthor{Adrian M. Price-Whelan}

\author[0000-0003-2866-9403]{David~W.~Hogg}
\affiliation{Max-Planck-Institut f\"ur Astronomie,
             K\"onigstuhl 17, D-69117 Heidelberg, Germany}
\affiliation{Center for Cosmology and Particle Physics,
             Department of Physics,
             New York University, 726 Broadway,
             New York, NY 10003, USA}
\affiliation{Center for Data Science,
             New York University, 60 Fifth Ave,
             New York, NY 10011, USA}
\affiliation{Flatiron Institute,
             Simons Foundation,
             162 Fifth Avenue,
             New York, NY 10010, USA}

\begin{abstract}\noindent % trust me
Action-space is \textsl{over}.
\end{abstract}

\keywords{Galaxy: kinematics and dynamics}

\section{Introduction}
\label{sec:intro}

Suh.

\section{Data}
\label{sec:data}

We use astrometric data from the \gaia\ mission (\citealt{Prusti:2016}), data
release 2 (\citealt{Gaia-Collaboration:2018, Lindegren:2018}).

% \begin{figure}
% \begin{center}
% \includegraphics[width=\textwidth]{gd1_sample.pdf}
% \end{center}
% \caption{
% On-sky positions of likely GD-1 members in the GD-1 coordinate system.
% GD-1 is apparent as an overdensity in negative proper motions (top right panel,
% orange box), so selecting on proper motion already reveals the stream in
% positions of individual stars (top left).
% The stream also stands out in the color-magnitude diagram (bottom right) as
% older and more metal poor than the background.
% Selecting the main sequence of GD-1 (orange, shaded region in bottom right)
% along with proper motion cuts unveils the stream in unprecedented detail (bottom
% left).
% }
% \label{fig:selection}
% \end{figure}

\section{Methods}
\label{sec:methods}

\begin{equation}
    v_i = v_{i, {\rm obs}} + A_{ij}\,x_j +
        B_{ijk}\,x_j\,v_k + C_{ijk}\,x_j\,x_k +
        D_{ijkm}\,x_j\,x_k\,v_m + F_{ijkm}\,x_j\,v_k\,v_m +
        G_{ijkm}\,x_j\,x_k\,x_m +
        ...
\end{equation}
where $C$ is symmetric in $j, k$, $D$ is symmetric in $j, k$, $F$ is symmetric
in $k, m$, and $G$ is symmetric in $j, k, m$.
I count: 9 + 27 + 18 + 54 + 54 + 30 parameters?

\section{Results}
\label{sec:results}

Huh.

\section{Discussion}
\label{sec:discussion}

Wuh?


\acknowledgements{
It is a pleasure to thank
Kathryn V. Johnston,
Robyn Sanderson,
Lauren Anderson,
and David N. Spergel for useful discussions and feedback.

This work has made use of data from the European Space Agency (ESA) mission {\it
Gaia} (\url{https://www.cosmos.esa.int/gaia}), processed by the {\it Gaia} Data
Processing and Analysis Consortium (DPAC,
\url{https://www.cosmos.esa.int/web/gaia/dpac/consortium}). Funding for the DPAC
has been provided by national institutions, in particular the institutions
participating in the {\it Gaia} Multilateral Agreement.

% KITP shit:
This work was ... KITP...
This research was supported in part by the National Science Foundation under
Grant No. NSF PHY-1748958.
}

\software{
    \package{Astropy} \citep{astropy, astropy:2018},
    % \package{dustmaps}\footnote{\url{https://github.com/gregreen/dustmaps}},
    \package{gala} \citep{gala},
    \package{IPython} \citep{ipython},
    \package{matplotlib} \citep{mpl},
    \package{numpy} \citep{numpy},
    \package{scipy} \citep{scipy}
}

\bibliographystyle{aasjournal}
\bibliography{myspace}

% \clearpage

% \appendix
% \section{Completeness and the \gaia\ scanning pattern}
% \label{sec:completeness}

% \figurename~\ref{fig:XX} (XX panel) shows the $V$-band extinction
% in the region around the GD-1 stream, computed from the
% Schlegel-Finkbeiner-Davis extinction map (\cite{Schlegel:1998}; hereafter SFD).

% % Notebook:
% \begin{figure}[h]
% \begin{center}
% \includegraphics[width=0.7\textwidth]{nvisits.pdf}
% \end{center}
% \caption{%
% TODO
% \label{fig:TODO}
% }
% \end{figure}


\end{document}
