% This file is part of the myspace project.
% Copyright 2019, 2020 the authors. All rights reserved.

% TODO:
% -

\documentclass[modern]{aastex62}
% Load common packages
% \usepackage{microtype}  % ALWAYS!
\usepackage{amsmath}
\usepackage{amsfonts}
\usepackage{amssymb}
\usepackage{booktabs}

\usepackage{graphicx}
\usepackage{color}

% \definecolor{cbblue}{HTML}{3182bd}
% \usepackage{hyperref}
% \definecolor{linkcolor}{rgb}{0.02,0.35,0.55}
% \definecolor{citecolor}{rgb}{0.45,0.45,0.45}
% \hypersetup{colorlinks=true,linkcolor=linkcolor,citecolor=citecolor,
%             filecolor=linkcolor,urlcolor=linkcolor}
% \hypersetup{pageanchor=true}

\newcommand{\documentname}{\textsl{Article}}
\newcommand{\sectionname}{Section}
\renewcommand{\figurename}{Figure}
\newcommand{\equationname}{Equation}
\renewcommand{\tablename}{Table}

% Packages / projects / programming
\newcommand{\package}[1]{\textsl{#1}}
\newcommand{\acronym}[1]{{\small{#1}}}
\newcommand{\github}{\package{GitHub}}
\newcommand{\python}{\package{Python}}
\newcommand{\emcee}{\package{emcee}}

% Missions
\newcommand{\project}[1]{\textsl{#1}}
\newcommand{\gaia}{\project{Gaia}}
\newcommand{\pans}{\project{Pan-STARRS}}

% For referee
\newcommand{\changes}[1]{{\color{red} #1}}
\definecolor{mahogany}{RGB}{165,15,21}
\newcommand{\resp}[1]{{\color{mahogany}#1}}

% Stats / probability
\newcommand{\given}{\,|\,}
\newcommand{\norm}{\mathcal{N}}

% Maths
\newcommand{\dd}{\mathrm{d}}
\newcommand{\transpose}[1]{{#1}^{\mathsf{T}}}
\newcommand{\inverse}[1]{{#1}^{-1}}
\newcommand{\argmin}{\operatornamewithlimits{argmin}}
\newcommand{\mean}[1]{\left< #1 \right>}

% Unit shortcuts
\newcommand{\msun}{\ensuremath{\mathrm{M}_\odot}}
\newcommand{\kms}{\ensuremath{\mathrm{km}~\mathrm{s}^{-1}}}
\newcommand{\mps}{\ensuremath{\mathrm{m}~\mathrm{s}^{-1}}}
\newcommand{\pc}{\ensuremath{\mathrm{pc}}}
\newcommand{\kpc}{\ensuremath{\mathrm{kpc}}}
\newcommand{\kmskpc}{\ensuremath{\mathrm{km}~\mathrm{s}^{-1}~\mathrm{kpc}^{-1}}}

% Misc.
\newcommand{\bs}[1]{\boldsymbol{#1}}

% Astronomy
\newcommand{\DM}{{\rm DM}}
\newcommand{\feh}{\ensuremath{{[{\rm Fe}/{\rm H}]}}}
\newcommand{\df}{\acronym{DF}}

% TO DO
\newcommand{\todo}[1]{{\color{red} TODO: #1}}


% typography
\setlength{\parindent}{1.\baselineskip}

% aastex parameters
% \received{not yet; THIS IS A DRAFT}
%\revised{not yet}
%\accepted{not yet}
% % Adds "Submitted to " the arguement.
% \submitjournal{ApJ}
\shorttitle{TODO}
\shortauthors{Price-Whelan \& Hogg}

\begin{document}\sloppy\sloppypar\raggedbottom\frenchspacing % trust me

\title{\textbf{\texttt{myspace}: action-space is \textsl{over}}}

% \author[0000-0003-0872-7098]{Adrian~M.~Price-Whelan}
\author{Adrian~M.~Price-Whelan}

% \author[0000-0003-2866-9403]{David~W.~Hogg}

\begin{abstract}\noindent % trust me
\end{abstract}

\keywords{Galaxy: kinematics and dynamics}

% \section{Introduction}
% \label{sec:intro}

% Suh.

% \section{Data}
% \label{sec:data}

% We use astrometric data from the \gaia\ mission (\citealt{Prusti:2016}), data
% release 2 (\citealt{Gaia-Collaboration:2018, Lindegren:2018}).

% \begin{figure}
% \begin{center}
% \includegraphics[width=\textwidth]{gd1_sample.pdf}
% \end{center}
% \caption{
% On-sky positions of likely GD-1 members in the GD-1 coordinate system.
% GD-1 is apparent as an overdensity in negative proper motions (top right panel,
% orange box), so selecting on proper motion already reveals the stream in
% positions of individual stars (top left).
% The stream also stands out in the color-magnitude diagram (bottom right) as
% older and more metal poor than the background.
% Selecting the main sequence of GD-1 (orange, shaded region in bottom right)
% along with proper motion cuts unveils the stream in unprecedented detail (bottom
% left).
% }
% \label{fig:selection}
% \end{figure}

\section{Methods}
\label{sec:methods}

Every star has an observed 3-vector velocity $\vec{v}$ and an observed
3-vector position $\vec{x}$.
We transform to a new transformed velocity $\vec{v}'$ according to
\begin{equation}
  \vec{v}' = \vec{v} + \mat{A}\cdot\vec{x} + \mat{B}(\vec{x})\cdot\vec{v} + \mbox{terms}
  ~,
\end{equation}
where
$\mat{A}$ is an arbitrary $3\times 3$ tensor and
$\mat{B}(\vec{x})$ is a position-dependent, unit-determinant $3\times 3$ tensor
of the following form:
\begin{equation}
  B(\vec{x}) = \exp\left(\sum_{k=1}^8 (\vec{u}_k^\top\cdot\vec{x})\,\mat{M}_k\right)
  ~,
\end{equation}
where
the exponentiation is \emph{matrix} exponentiation,
the 8 3-vectors $\vec{u}_k$ are free parameters,
and the 8 matrices $\mat{M}_k$ are the following traceless rotation and shear generators (or any 8 independent linear combinations thereof):
\begin{equation}
  \begin{bmatrix} 0 & 1 & 0 \\ -1 & 0 & 0 \\ 0 & 0 & 0 \end{bmatrix}
  \begin{bmatrix} 0 & 0 & -1 \\ 0 & 0 & 0 \\ 1 & 0 & 0 \end{bmatrix}
  \begin{bmatrix} 0 & 0 & 0 \\ 0 & 0 & 1 \\ 0 & -1 & 0 \end{bmatrix}
\end{equation}
\begin{equation}
  \begin{bmatrix} 1 & 0 & 0 \\ 0 & -1 & 0 \\ 0 & 0 & 0 \end{bmatrix}
  \begin{bmatrix} -1 & 0 & 0 \\ 0 & 0 & 0 \\ 0 & 0 & 1 \end{bmatrix}
  \begin{bmatrix} 0 & 1 & 0 \\ 1 & 0 & 0 \\ 0 & 0 & 0 \end{bmatrix}
  \begin{bmatrix} 0 & 0 & 1 \\ 0 & 0 & 0 \\ 1 & 0 & 0 \end{bmatrix}
  \begin{bmatrix} 0 & 0 & 0 \\ 0 & 0 & 1 \\ 0 & 1 & 0 \end{bmatrix}
\end{equation}
This model has 9 free parameters in $\mat{A}$ and 24 total across
the 8 vectors $\vec{u}_k$, making for 33 free parameters in total.

In case you have ``forgotten'', matrix exponentiation can be defined as
\begin{equation}
  \exp(\mat{Q}) = I + \mat{Q} + \frac{1}{2!}\,\mat{Q}\cdot\mat{Q}
  + \frac{1}{3!}\,\mat{Q}\cdot\mat{Q}\cdot\mat{Q}
  + \frac{1}{4!}\,\mat{Q}\cdot\mat{Q}\cdot\mat{Q}\cdot\mat{Q}
  + \cdots
  ~,
\end{equation}
or you can just use \code{scipy.linalg.expm()} and avoid thinking about it.

Our objective is the log probability of the suburbs stars given then
GMM fit to the city stars. Say lots more about all that.

We previously imagined doing this:
\begin{equation}
    v_i = v_{i, {\rm obs}} + A_{ij}\,x_j +
        B_{ijl}\,x_j\,v_l + C_{ijl}\,x_j\,x_l +
        D_{ijlm}\,x_j\,x_l\,v_m + F_{ijlm}\,x_j\,v_l\,v_m +
        G_{ijlm}\,x_j\,x_l\,x_m +
        ...
\end{equation}
where $C$ is symmetric in $j, l$, $D$ is symmetric in $j, l$, $F$ is symmetric
in $l, m$, and $G$ is symmetric in $j, l, m$.
I count: 9 + 27 + 18 + 54 + 54 + 30 parameters?
But this is perhaps unstable, because the transformation can collapse the
velocity distribution into a tiny space.


\section{Results}
\label{sec:results}

Huh.

\section{Discussion}
\label{sec:discussion}

Wuh?


\acknowledgements{
It is a pleasure to thank
Kathryn V. Johnston,
Robyn Sanderson,
Lauren Anderson,
and David N. Spergel for useful discussions and feedback.

This work has made use of data from the European Space Agency (ESA) mission {\it
Gaia} (\url{https://www.cosmos.esa.int/gaia}), processed by the {\it Gaia} Data
Processing and Analysis Consortium (DPAC,
\url{https://www.cosmos.esa.int/web/gaia/dpac/consortium}). Funding for the DPAC
has been provided by national institutions, in particular the institutions
participating in the {\it Gaia} Multilateral Agreement.

% KITP shit:
This work was ... KITP...
This research was supported in part by the National Science Foundation under
Grant No. NSF PHY-1748958.
}

\software{
    \package{Astropy} \citep{astropy, astropy:2018},
    % \package{dustmaps}\footnote{\url{https://github.com/gregreen/dustmaps}},
    \package{gala} \citep{gala},
    \package{IPython} \citep{ipython},
    \package{matplotlib} \citep{mpl},
    \package{numpy} \citep{numpy},
    \package{scipy} \citep{scipy}
}

\bibliographystyle{aasjournal}
\bibliography{myspace}

% \clearpage

% \appendix
% \section{Completeness and the \gaia\ scanning pattern}
% \label{sec:completeness}

% \figurename~\ref{fig:XX} (XX panel) shows the $V$-band extinction
% in the region around the GD-1 stream, computed from the
% Schlegel-Finkbeiner-Davis extinction map (\cite{Schlegel:1998}; hereafter SFD).

% % Notebook:
% \begin{figure}[h]
% \begin{center}
% \includegraphics[width=0.7\textwidth]{nvisits.pdf}
% \end{center}
% \caption{%
% TODO
% \label{fig:TODO}
% }
% \end{figure}


\end{document}
